\documentclass[11pt]{article}
%%%%可按下 Ctrl + /(Windows/Linux)或 Cmd + /(Mac)批量注释或取消注释(即在预览PDF中不显示或显示该部分)    
\usepackage{hyperref}
\usepackage{xcolor}
\usepackage{calc}
\usepackage{graphicx}
\usepackage{tikz}
\usepackage{fontspec}
\usepackage{fontawesome5}
\usepackage{titlesec}
\usepackage{enumitem}
\usepackage{fancybox}
\usepackage{tabularx}
\hypersetup{hidelinks}

\setlength{\parindent}{0pt}					% 取消全局段落缩进
\pagenumbering{gobble}						% 取消页码显示
\setlist[itemize]{nosep                     % 取消 itemize 的默认间距
    , before={\vspace*{-\parskip}}          % 取消 itemize 和后续段落之间的空白
    , leftmargin=*}		                    % 取消 itemize 的左边距
\setlist[enumerate]{leftmargin=*}	        % 取消 enumerate 的左边距
\renewcommand{\arraystretch}{1.2}           % 设置表格行间距
\linespread{1.25}                           % 设置正文行间距

\titleformat{\section}					    % 将原标题前面的数字取消了
  {\LARGE\bfseries\raggedright} 		      % 字体改为 LARGE,bold,左对齐
  {}{0em}                      			  % 可用于添加全局标题前缀
  {}                           			  % 可用于添加代码
  [{\color{mygray}\titlerule}]               % 标题下方加一条线,使用自定义的灰色 
\titlespacing*{\section}{0cm}{*1.2}{*1.2}	% 标题左边留白,上方,下方

\usepackage[
	a4paper,
	left=1.2cm,
	right=1.2cm,
	top=1.5cm,
	bottom=1cm,
	nohead
]{geometry}                                 % 页面边距设置

% 字体设置
\setmainfont[
    Path=fonts/,
    Extension=.otf,
    BoldFont=*-Bold,
]{NotoSerifSC}

% 定义颜色(中大绿)
\definecolor{SYSU_Green}{RGB}{0,102,0} 
\definecolor{mygray}{RGB}{128,128,128} 
\newlength{\iconwidth}
\setlength{\iconwidth}{1.5em}                   % 设置 section 标题部分图标占用的宽度



% 学院
\newcommand{\school}{XXX学院 | School of ABC} 

% 联系方式
\newcommand{\contact}{
    % 根据个人喜好选择字号
    % \small                % 小
    \footnotesize           % 更小
    % \scriptsize           % 再小一号
    \textcolor{white}{
        % 邮箱
        \faEnvelope \quad \href{mailto:youremail@mail2.sysu.edu.cn}{youremail@mail2.sysu.edu.com}
        \hspace{4em}
        % 手机号
        \faPhone \quad  123-4567-8900(微信同号)
        % 别的联系方式,如微信、GitHub等
        \hspace{4em}
        \faGithub \quad \href{https://github.com/////}{GitHub}
    }
}

\begin{document}

    %%%%%%%%%%%%%%%%%%%%
    % 页眉、页脚和背景(如果有多页简历,请把页眉页脚和背景复制粘贴到第二页的内容之前)
    %%%%%%%%%%%%%%%%%%%%

    % 页眉:校标组合+学院名
    \begin{tikzpicture}[remember picture, overlay]
        \node[anchor=north, inner sep=0pt](header) at (current page.north){
            \includegraphics[width=\paperwidth]{images/header_sysu.png}
        };
        \node[anchor=west](school_logo) at (header.west){
            \hspace{0.5cm}
            \includegraphics[width=0.2\textwidth]{images/SYSU_smallwhitelogo.png}
        };
        \node[anchor=east](school_name) at(header.east){
            \textcolor{white}{\textbf{\school}}
            \hspace{0.5cm}
        };
    \end{tikzpicture}
    \vspace{-3.5em}

    % 页脚,联系方式
    \begin{tikzpicture}[remember picture, overlay]
        \node[anchor=south, inner sep=0pt](footer) at (current page.south){
            \includegraphics[width=\paperwidth]{images/footer_sysu.png}
        };
        % 联系方式
        \node[anchor=center] at(footer.center){\contact};
    \end{tikzpicture}

    % 背景
    \begin{tikzpicture}[remember picture, overlay]
        \node[opacity=0.05] at(current page.center){
            \includegraphics[width=0.7\paperwidth, keepaspectratio]{images/SYSU_big_logo.png}
        };
    \end{tikzpicture}

    %%%%%%%%%%%%%%%%%%%%
    % 简历正文
    %%%%%%%%%%%%%%%%%%%%

    \begin{minipage}[t]{0.78\textwidth}
        % 个人信息
        \begin{minipage}[t]{\textwidth}
        \section[个人信息]{\makebox[\iconwidth][c]{\color{SYSU_Green}{\faAddressCard}}\quad 个人信息}
        \begin{minipage}[t]{0.5\textwidth}
            \textbf{姓\qquad 名}:你的名字
            
            \vspace{0.5em}
            \textbf{出生年月}:你的出生日期
        \end{minipage}
        \begin{minipage}[t]{0.35\textwidth}
            \textbf{性\qquad 别}:random
            
            \vspace{0.5em}
            \textbf{政治面貌}:你的政治面貌
        \end{minipage}
        \vspace{1.2em}
        \end{minipage}

        % 教育背景
        \begin{minipage}[t]{\textwidth}
        \section[教育背景]{\makebox[\iconwidth][c]{\color{SYSU_Green}{\faGraduationCap}}\quad 教育背景}
        
        {\large \textbf{中山大学}} 本科 \hfill 2022年9月--2026年6月
        \begin{itemize}
            \item 你的学院,你的专业
            \item \textbf{主修课程}:课程1、课程2、课程3、课程4\ 等。
            \item \textbf{GPA}:4.8 / 5(排名:1 / 250)%自行考虑是否写课程和GPA
        \end{itemize}
        
        \vspace{0.5em}
        {\large \textbf{中山大学}} 硕士 \hfill 2026年9月--2029年6月
        \begin{itemize}
            \item 你的学院,你的专业
            \item \textbf{主修课程}:课程1、课程2、课程3、课程4\ 等。
            \item \textbf{GPA}:4.8 / 5(排名:1 / 250)%自行考虑是否写课程和GPA
            \item 你的学院,你的专业,你的导师姓名\ 职称             
            \item \textbf{研究方向}:方向1、方向2、方向3、方向4\ 等。
        \end{itemize}
        
        \vspace{0.5em}
        {\large \textbf{中山大学}} 博士 \hfill 2029年9月--未知
        \begin{itemize}
            \item 你的学院,你的专业,你的导师姓名\ 职称
            \item \textbf{研究方向}:方向1、方向2、方向3、方向4\ 等。
        \end{itemize}
        
        \vspace{1.2em}
        \end{minipage}
    \end{minipage}
    \hfill
    % 右半边,照片,比例占行宽20%
    \begin{minipage}[t]{0.2\textwidth}
        \vspace{2em} % 照片上侧内容
        \setlength{\fboxsep}{0pt}
        \doublebox{\includegraphics[width=\linewidth]{images/avatar.png}}
    \end{minipage}

    \begin{minipage}[t]{\textwidth}
    % 科研成果
    \section[科研成果]{\makebox[\iconwidth][c]{\color{SYSU_Green}{\faAtom}}\quad 科研成果}

    % 科研著作(研究生)
    This is One of Your Paper Published in Conference A.
    \begin{itemize}
        \item \textbf{yourname}, mentorname. \hfill 发表于 \textbf{Conference A}(CCF-A类会议)
        \item 这份论文干了什么\dots
    \end{itemize}

    \vspace{0.5em}
    《另一份论文的标题》
    \begin{itemize}
        \item  \textbf{你的名字}、你的同门、你导 \hfill 发表于 \textbf{某篇期刊} (SCI-1区)
        \item 这份论文干了什么\dots
    \end{itemize}
    
    \vspace{1.2em}
    \end{minipage}

    \begin{minipage}[t]{\textwidth}
    % 项目经历\科研经历\项目与教学(标题请根据需要修改)
    \section[项目与教学]{\makebox[\iconwidth][c]{\color{SYSU_Green}{\faChalkboardTeacher}}\quad 项目与教学}
    
    {\large \textbf{项目名称}} \hfill 2020年9月--2021年9月
    \begin{itemize}
        \item \textbf{你在项目中扮演的角色} \hfill 横向/纵向项目-已完结/进行中
        \item 介绍你在这个项目中的具体工作内容\dots\dots
    \end{itemize}

%     \vspace{0.5em}
%     {\large \textbf{某某主题讨论班}},主讲 / 参与 \hfill 2020年夏季
%     \begin{itemize}
%         \item \textbf{主要内容}:内容1,内容2,内容3\ 等。
%     \end{itemize}

%     \vspace{0.5em}
%     {\large \textbf{课程名称}},助教 \hfill 2021年夏季
%     \begin{itemize}
%         \item \textbf{主要内容}:内容1,内容2,内容3\ 等。
%     \end{itemize}
%     % 支教经历
%     {\large \textbf{某地支教项目}} \hfill 2021年7月--2021年8月
%     \begin{itemize}
%     \item \textbf{角色}:支教老师 \hfill 志愿项目
%     \item \textbf{内容}:为当地学生教授数学和英语课程,组织课外活动,提升学生的学习兴趣。
%     \end{itemize}

%     % 志愿服务
%     {\large \textbf{某社区志愿服务}} \hfill 2020年3月--2020年5月
%     \begin{itemize}
%     \item \textbf{角色}:志愿者 \hfill 社区服务
%     \item \textbf{内容}:协助社区防疫工作,分发物资,为居民提供生活帮助。
% \end{itemize}
    \vspace{1.2em}
    \end{minipage}
\newpage

    %%%%%%%%%%%%%%%%%%%%
    % 页眉、页脚和背景(如果有多页简历,请把页眉页脚和背景复制粘贴到第二页的内容之前)
    %%%%%%%%%%%%%%%%%%%%

    % 页眉:校标组合+学院名
\begin{tikzpicture}[remember picture, overlay]
    \node[anchor=north, inner sep=0pt](header) at (current page.north){
        \includegraphics[width=\paperwidth]{images/header_sysu.png}
    };
    \node[anchor=west](school_logo) at (header.west){
        \hspace{0.5cm}
        \includegraphics[width=0.2\textwidth]{images/SYSU_smallwhitelogo.png}
        };
    \node[anchor=east](school_name) at(header.east){
        \textcolor{white}{\textbf{\school}}
        \hspace{0.5cm}
    };
\end{tikzpicture}
\vspace{-3.5em}

% 页脚,联系方式
\begin{tikzpicture}[remember picture, overlay]
    \node[anchor=south, inner sep=0pt](footer) at (current page.south){
        \includegraphics[width=\paperwidth]{images/footer_sysu.png}
    };
    % 联系方式
    \node[anchor=center] at(footer.center){\contact};
\end{tikzpicture}

% 背景
\begin{tikzpicture}[remember picture, overlay]
    \node[opacity=0.05] at(current page.center){
        \includegraphics[width=0.7\paperwidth, keepaspectratio]{images/SYSU_big_logo.png}
    };
\end{tikzpicture}
% 如果每行的内容不是很多,可以考虑使用 minipage,将内容分列展示
\begin{minipage}[t]{0.6\textwidth}
        \section[技能特长]{\makebox[\iconwidth][c]{\color{SYSU_Green}{\faWrench}}\quad 技能特长}
        \begin{itemize}
        \setlength{\itemsep}{0.5em}
            \item 熟练使用 Python、C++ 等编程语言。
            \item 熟练使用 Tensorflow、Pytorch 等深度学习框架。
            \item 熟悉 Windows 与 Linux 端开发。
        \end{itemize}
    \end{minipage}
    \hfill
    \begin{minipage}[t]{0.35\textwidth}
        \section[兴趣爱好]{\makebox[\iconwidth][c]{\color{SYSU_Green}{\faStar}}\quad 兴趣爱好}
        \begin{itemize}
        \setlength{\itemsep}{0.5em}
            \item SING
            \item DANCE
            \item Rap
            \item BASKETBALL
        \end{itemize}
    \end{minipage}
    
    % \newpage
    % % 如有需要,可以添加额外的页面。不要忘记添加页眉页脚和背景相关的代码。

    % 竞赛经历
    \section{\makebox[\widthof{\faTrophy}][c]{\color{SYSU_Green}{\faTrophy}}\quad 竞赛经历}
    \begin{table}[h!]
        \begin{tabularx}{\textwidth}{Xp{\widthof{第几负责人}}p{\widthof{国家级-第100名}}p{\widthof{2025年1月}}}
            \textbf{比赛1} & 第一负责人 & 国家级-第10名 & 2023年4月 \\
            \textbf{比赛2} & 个人参赛 & 国家级-一等奖 & 2023年8月\\
            \textbf{比赛3} & 个人参赛 & 省级-一等奖 & 2022年12月\\
            % 同理,可以自己加
        \end{tabularx}
    \end{table}

    % % 技能特长
    % \section{\makebox[\widthof{\faWrench}][c]{\color{SYSU_Green}{\faWrench}}\quad 技能特长}
    % \begin{itemize}
    %     \item 熟练使用C++ 、Python、Matlab编程语言。
    %     \item 熟悉Windows与Linux端开发。
    %     \item 熟练使用Tensorflow,Pytorch等深度学习框架。
    %     \item 熟练使用Keil,Arduino IDE等集成开发软件。
    %     \item 了解模式识别,强化学习,遗传算法,知识蒸馏等相关概念。
    % \end{itemize}

% 所获荣誉
\section{\makebox[\widthof{\faStar}][c]{\color{SYSU_Green}{\faStar}}\quad 所获荣誉}
\begin{minipage}[t]{0.48\textwidth}
    \begin{itemize}
        \item 某年学业先进个人
        \item 某年某奖学金某等奖
        \item 某年优秀团员称号
    \end{itemize}
\end{minipage}
\hfill
\begin{minipage}[t]{0.48\textwidth}
    \begin{itemize}
        \item 某大使
        \item 某年某奖学金某等奖
        \item 某年某称号
    \end{itemize}
\end{minipage}

    % 其他
    \section{\makebox[\widthof{\faInfo}][c]{\color{SYSU_Green}{\faInfo}}\quad 其他}
    \begin{itemize}
        \item 英语水平-CET6/雅思/TOEFL xxx分
        \item xx几级证书
        \item 技术博客: 某网址
        \item 教师资格证:xxx
        \item 普通话证书:几级几等
        \item 文字排版:\LaTeX
    \end{itemize}

\end{document}
